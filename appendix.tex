\appendix
\section*{ضمیمه A: شبیه‌سازی پروتکل‌های معکوس زمانی با Qiskit}
\label{sec:qiskit-appendix}

در این بخش نشان داده می‌شود که پروتکل‌های معکوس زمانی مورد استفاده در پژوهش حاضر را می‌توان به‌صورت عددی در محیط \lr{Qiskit} شبیه‌سازی کرد. این شبیه‌سازی به پژوهشگران اجازه می‌دهد تا پویایی همبستگی‌های خارج از ترتیب زمانی را در سامانه‌های کوچک بررسی و تحلیل کنند.
در این رویکرد، دو بلوک زمانی متناظر با تکامل روبه‌جلو و معکوس زمان به‌صورت زیر تعریف می‌شوند:
\begin{center}
	\begin{LTR}
		\begin{minted}[frame=lines,bgcolor=gray!10,fontsize=\scriptsize,linenos=false]{python}
from qiskit import QuantumCircuit
from qiskit.quantum_info import random_unitary

U = random_unitary(4)     # operator on 2 qubits
qc = QuantumCircuit(2)
qc.unitary(U, [0, 1])           # forward evolution
qc.barrier()
qc.unitary(U.adjoint(), [0, 1]) # time inverted evolution
qc.draw('mpl')
\end{minted}
\end{LTR}
\end{center}
ماژول \lr{qiskit-dynamics} امکان تعریف همیلتونی و محاسبه‌ی مستقیم اپراتور \( U(t)=e^{-iHt} \) را فراهم می‌کند.  
در ادامه می‌توان مقادیر انتظاری OTOC را به‌صورت عددی محاسبه کرد:
\begin{center}
	\begin{LTR}
		\begin{minted}[frame=lines,bgcolor=gray!10,fontsize=\scriptsize,linenos=false]{python}
from qiskit.quantum_info import Statevector, Pauli

psi0 = Statevector.from_label('00')
M = Pauli('Z')
U_t = U
U_t_dag = U_t.adjoint()
# calculate  ⟨M(t) M⟩ as simple OTOC: 
psi_t = psi0.evolve(U_t)
otoc = psi_t.expectation_value(M)
\end{minted}
\end{LTR}
\end{center}

به‌کمک این ابزارها می‌توان:
\begin{itemize}
	\item تأثیر اپراتورهای پائولی تصادفی درج‌شده در میانه‌ی تکامل را بررسی کرد؛
	\item وابستگی OTOC به زمان یا پارامترهای همیلتونی را مطالعه نمود؛
	\item و حتی رفتار تداخل‌های چندمسیره‌ی مشابه با نتایج مقاله را در مقیاس‌های کوچک بازسازی کرد.
\end{itemize}

این روش شبیه‌سازی می‌تواند در حوزه‌ی \lr{QNLP} نیز مورد استفاده قرار گیرد؛ به‌گونه‌ای که مدارهای زبانی \lr{lambeq} با افزودن بلوک‌های معکوس زمانی، برای تحلیل \textbf{تداخل‌های زمانی–معنایی} در جمله‌ها به‌کار روند. این کار راهی نو برای بررسی پویایی معنای کوانتومی در گذر از ساختارهای نحوی مختلف فراهم می‌سازد.
